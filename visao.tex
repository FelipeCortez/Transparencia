\documentclass[12pt, a4paper]{article}
\usepackage[utf8]{inputenc}
\usepackage[brazilian]{babel} % Hifenização e dicionário
\usepackage{enumitem}
\usepackage{multirow}
\usepackage[left=3.00cm, right=2.00cm, top=3.00cm, bottom=2.00cm]{geometry}

\begin{document}

    \begin{center}
      \textsc{UFRN --- Universidade Federal do Rio Grande do Norte} \\
      \textsc{DIMAp --- Departamento de Informática e Matemática Aplicada} \\
    \end{center}

    \bigskip

    \begin{tabular}{@{}ll@{}}
        \emph{Disciplina:} & DIM0XXX --- Engenharia de Software \\
        \emph{Docente:}    & Márcia Jacyntha \\
        \emph{Discentes:}  & Felipe Cortez de Sá \small{(2012912357)} \\
                           &  \small{(2012xxxxxx)}\\
                           & Luiz Pablo \small{(2012xxxxxx)}
    \end{tabular}

    \bigskip

    \begin{center}
      \Large\textbf{Documento de Visão}
    \end{center}

    \section{Introdução}
    \subsection{Finalidade}
    \subsection{Escopo do documento}
    \subsection{Definições, acrônimos e abreviações}
    \subsection{Referências}

    \section{Contextualização}
    \subsection{Descrição do problema}
    \subsection{Sentença de posição do produto}

    \section{Descrição dos stakeholders e dos usuários}
    \subsection{Principais stakeholders e usuários}
    \subsection{Necessidades chave dos stakeholders e dos usuários}

    \section{Visão geral do produto}
    \subsection{Perspectiva do produto}
    \subsection{Premissas e dependências}
    \subsection{Limites do produto}

    \section{Requisitos funcionais do produto}

    \section{Requisitos não funcionais do produto}

    \section{Restrições técnicas}

\end{document}
